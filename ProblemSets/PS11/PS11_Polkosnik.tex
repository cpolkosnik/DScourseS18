\documentclass{article}
\usepackage[utf8]{inputenc}
\usepackage[authoryear]{natbib}
\title{PS11}
\author{Conrad Polkosnik }
\date{April 2018}

\usepackage{natbib}
\usepackage{graphicx}

\begin{document}

\maketitle

\section{Introduction}

\begin{description}
\item What is the stock market?
\item Why is it important?
\item Who can be a part of the stock market? Who is involved in the markets?
\item Trading and investing
\end{description}

\section{Literature Review}

\begin{description}
\item History/Background of stock price prediction
\item What have poeple done to predict stock prices?
\item What techniques have they used?
\item Did these techniques work?
\item What methods were good? What methods were bad?
\item What is technical analysis? How is it used?
\item Implications of stock prediction
\end{description}

\section{Data}

\begin{description}
\item Where did I get the data from?
\item Is the data trustworthy/unbiased?
\item What data did I use specifically?
\item Differences in prediction based on different time periods of historical data
\item Differences in prediction based on number of simulations
\end{description}

\section{Methods}

\begin{description}
\item Quantmod for data
\item Visualization of overall trend
\item Fundamental Technical Analysis
\item Analyzation of distribution (log values, mean, standard deviation)
\item Random Walk
\item Monte Carlo Simulation
\end{description}

\section{Findings}

\begin{description}
\item Technical Analysis
\item Log returns
\item Probabilities and likelihood
\item Daily average increase
\item Random Walk (parameters, organization, visualization)
\item Monte Carlo Simulation
\end{description}

\section{Conclusions}

\begin{description}
\item Summarize overall introduction and literature review in regard to the benefits from doing this project.
\item Summarize data, methods, and findings.
\item What was the ultimate goal of the project? Was it met? Does it make sense?
\end{description}
\bibliographystyle{jpe}
\nocite{*}
\bibliography{sample.bib}

\section{Code}
\begin{verbatim}
---
title: "Final Project"
author: "Conrad Polkosnik"
date: "3/6/2018"
output: word_document
---

Loading Packages
```{r}
library(quantmod)   # get stock prices & useful stock analysis functions
library(xts)        # extensible time series
library(tidyverse)  # ggplot2, purrr, dplyr, tidyr, readr, tibble
library(stringr)    # working with strings
```


Downloading Stock Data
```{r}
# Calling quantmod
library("quantmod")

# Gathering historical stock data for Exxon Mobil since 1970
getSymbols("XOM", src='yahoo', from="1970-01-01")

```


```{r}
# Understanding data pulled
library(magrittr) #for pipe operator

XOM %>% class()

XOM %>% str()

XOM %>% head()
```

Visualize Trend
```{r}
XOM %>% Ad() %>% chartSeries()
```


Fundamental Technical Analysis
```{r}
XOM %>% chartSeries(TA='addBBands();
                    addBBands(draw="p");
                    addVo();
                    addMACD()',
                    subset='1970-2018',
                    theme="white"
                    )
```


Gathering log daily returns for XOM
```{r}
XOM %>% Ad() %>% dailyReturn(type = 'log') %>% head()
```

Using log returns to analyze distribution
```{r}
XOMLogReturns <- XOM %>% Ad() %>% dailyReturn(type = "log")

names(XOMLogReturns) <- "XOM.Log.Returns"

# Plotting the log returns using a histogram
library(ggplot2)
XOMLogReturns %>% ggplot(aes(x = XOM.Log.Returns)) + geom_histogram(bins = 100) + geom_density() + geom_rug(alpha = 0.5)
```

Setting probabilities for study of liklihood
```{r}
Probabilities <- c(.005, .025, .25, .5, .75, .975, .995)

DistributionalLogReturns <- XOMLogReturns %>% quantile(probs = Probabilities, na.rm = TRUE)

DistributionalLogReturns
```


```{r}
MeanLogReturns <- mean(XOMLogReturns, na.rm = TRUE)
StDevLogReturns <- sd(XOMLogReturns, na.rm = TRUE)

MeanLogReturns
StDevLogReturns
```

Finding daily average increase or decrease in return
```{r}
MeanLogReturns %>% exp()
```
Analysis: On average, the investor should expect 0.0657% increase daily in return.


Setting Parameters for Random Walk
```{r}
# Parameters
N <- 252
mu <- MeanLogReturns
sigma <- StDevLogReturns
day <- 1:N

InitialPrices <- XOM$XOM.Adjusted[[nrow(XOM$XOM.Adjusted)]]

set.seed(386) # for reproducibility
Price  <- c(InitialPrices, rep(NA, N-1))
for(i in 2:N) {
    Price[i] <- Price[i-1] * exp(rnorm(1, mu, sigma))
              }

library(tidyverse)
PriceSimulation <- cbind(day, Price) %>% as_tibble()

# Visualize price simulation
PriceSimulation %>%
    ggplot(aes(day, Price)) +
    geom_line() +
    ggtitle(str_c("XOM: Simulated Prices for ", N," Trading Days"))
```

Monte Carlo Simulation for Random Walk
```{r}
# Parameters
N     <- 252 # Number of Stock Price Simulations
M     <- 100  # Number of Monte Carlo Simulations
mu    <- MeanLogReturns
sigma <- StDevLogReturns
day <- 1:N
InitialPrices <- XOM$XOM.Adjusted[[nrow(XOM$XOM.Adjusted)]]

# Simulate prices using for loop and Monte Carlo Matrix
set.seed(386)
MonteCarloMatrix <- matrix(nrow = N, ncol = M)
for (j in 1:M) {
    MonteCarloMatrix[[1, j]] <- InitialPrices
    for(i in 2:N) {
        MonteCarloMatrix[[i, j]] <- MonteCarloMatrix[[i - 1, j]] * exp(rnorm(1, mu, sigma))
                  }
               }

# Organizing fata frame and format
PriceSimulation <- cbind(day, MonteCarloMatrix) %>% as_tibble()
nm <- str_c("Sim.", seq(1, M))
nm <- c("Day", nm)
names(PriceSimulation) <- nm
PriceSimulation <- PriceSimulation %>%
    gather(key = "Simulation", value = "Stock.Price", -(Day))

# Visualization
PriceSimulation %>%
    ggplot(aes(x = Day, y = Stock.Price, Group = Simulation)) +
    geom_line(alpha = 0.1) +
    ggtitle(str_c("XOM: ", M,
                  " Monte Carlo Simulations for Prices Over ", N,
                  " Trading Days"))
```

Stock price estimation and likelihood
```{r}
EstimatedStockPrices <- PriceSimulation %>% filter(Day == max(Day))

Probabilities <- c(.005, .025, .25, .5, .75, .975, .995)

DistributionalEstimatedStockPrices <- quantile(EstimatedStockPrices$Stock.Price, probs = Probabilities)

DistributionalEstimatedStockPrices %>% round(2)
```

\end{verbatim}

\end{document}
